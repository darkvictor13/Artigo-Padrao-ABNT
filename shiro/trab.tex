\documentclass[
	% -- opções da classe memoir --
	article,			% indica que é um artigo acadêmico
	12pt,				% tamanho da fonte
	oneside,			% para impressão apenas no recto. Oposto a twoside
	a4paper,			% tamanho do papel. 
	% -- opções da classe abntex2 --
	%chapter=TITLE,		% títulos de capítulos convertidos em letras maiúsculas
	%section=TITLE,		% títulos de seções convertidos em letras maiúsculas
	%subsection=TITLE,	% títulos de subseções convertidos em letras maiúsculas
	%subsubsection=TITLE % títulos de subsubseções convertidos em letras maiúsculas
	% -- opções do pacote babel --
	english,			% idioma adicional para hifenização
	brazil,				% o último idioma é o principal do documento
	sumario=tradicional
	]{abntex2}
% ---
% PACOTES
% ---

% ---
% Pacotes fundamentais 
% ---
\usepackage{lmodern}			% Usa a fonte Latin Modern
\usepackage[T1]{fontenc}		% Selecao de codigos de fonte.
\usepackage[utf8]{inputenc}		% Codificacao do documento (conversão automática dos acentos)
\usepackage{indentfirst}		% Indenta o primeiro parágrafo de cada seção.
\usepackage{nomencl} 			% Lista de simbolos
\usepackage{color}				% Controle das cores
\usepackage{graphicx}			% Inclusão de gráficos
\usepackage{microtype} 			% para melhorias de justificação
\usepackage{gensymb}

% ---
% Pacotes adicionais, usados apenas no âmbito do Modelo Canônico do abnteX2
% ---
% ---
% Pacotes de citações
% ---
\usepackage[brazilian,hyperpageref]{backref}	 % Paginas com as citações na bibl
%\usepackage[alf]{abntex2cite}	% Citações padrão ABNT
\usepackage[num,overcite,abnt-emphasize=bf]{abntex2cite}	% Citações padrão ABNT
%\usepackage[num]{abntex2cite}	% Citações padrão ABNT
%\citebrackets()
\citebrackets[]
% ---

% ---
% Configurações do pacote backref
% Usado sem a opção hyperpageref de backref
\renewcommand{\backrefpagesname}{Citado na(s) página(s):~}
% Texto padrão antes do número das páginas
\renewcommand{\backref}{}
% Define os textos da citação
\renewcommand*{\backrefalt}[4]{
	\ifcase #1 %
		Nenhuma citação no texto.%
	\or
		Citado na página #2.%
	\else
		Citado nas páginas #2.%
	\fi}%
% ---

% --- Informações de dados para CAPA e FOLHA DE ROSTO ---
% \titulo{Modelo Canônico de Artigo científico com \abnTeX}
% \tituloestrangeiro{Canonical article template in \abnTeX: optional foreign title}

\titulo{Sistemas Embarcados, aplicações na agricultura IOT: revisão bibliográfica}
\tituloestrangeiro{Use of free software for Computer Science students}

\autor{Victor Emanuel Almeida}

\local{FOZ DO IGUAÇU}
\data{\today}

%\instituicao{%
	%\par
	%Universidade do Oeste do Paraná 
	%\par
	%UNIOESTE
%}
\instituicao{Curso de Ciência da Computação, da Universidade Estadual do Oeste do Paraná (UNIOESTE), Campus Foz do Iguaçu-PR, Brasil}

\preambulo{Escrever Preâmbulo}

\tipotrabalho{Trabalho Acadêmico}
% ---

% ---
% Configurações de aparência do PDF final

% alterando o aspecto da cor azul
\definecolor{blue}{RGB}{41,5,195}

% informações do PDF
\makeatletter
\hypersetup{
  %pagebackref=true,
  pdftitle={\@title}, 
  pdfauthor={\@author},
  pdfsubject={software livre},
  pdfcreator={\@author},
  pdfkeywords={software livre},
  colorlinks=true,       		% false: boxed links; true: colored links
  linkcolor=black,          	% color of internal links
  citecolor=blue,        		% color of links to bibliography
  filecolor=magenta,      		% color of file links
  urlcolor=blue,
  bookmarksdepth=4
}
\makeatother
% --- 

% ---
% compila o indice
% ---
\makeindex
% ---

% ---
% Altera as margens padrões
% ---
\setlrmarginsandblock{3cm}{2cm}{*}
\setulmarginsandblock{3cm}{2cm}{*}
\checkandfixthelayout
% ---

% --- 
% Espaçamentos entre linhas e parágrafos 
% --- 

% O tamanho do parágrafo é dado por:
\setlength{\parindent}{1.25cm}
%\setlength{\parindent}{1.5\lineheight}

% Controle do espaçamento entre um parágrafo e outro:
%\setlength{\parskip}{0.2cm}  % tente também \onelineskip

% Espaçamento simples
\SingleSpacing


% ----
% Início do documento
% ----
\begin{document}
%\pagenumbering{roman}
% Seleciona o idioma do documento (conforme pacotes do babel)
%\selectlanguage{english}
\selectlanguage{brazil}

% Retira espaço extra obsoleto entre as frases.
\frenchspacing 

% ----------------------------------------------------------
% ELEMENTOS PRÉ-TEXTUAIS
% ----------------------------------------------------------

%---
%
% Se desejar escrever o artigo em duas colunas, descomente a linha abaixo
% e a linha com o texto ``FIM DE ARTIGO EM DUAS COLUNAS''.
%\twocolumn[    		% INICIO DE ARTIGO EM DUAS COLUNAS
%
%---

% página de titulo principal (obrigatório)
\imprimircapa
%\begin{center}
  %\maketitle
  %\imprimirtitulo
  %\vspace{.5cm}

  %\imprimirinstituicao
%\end{center}

% titulo em outro idioma (opcional)

% resumo em português
\begin{resumoumacoluna}
 Conforme a ABNT NBR 6022:2018, o resumo no idioma do documento é elemento obrigatório. 
 Constituído de uma sequência de frases concisas e objetivas e não de uma 
 simples enumeração de tópicos, não ultrapassando 250 palavras, seguido, logo 
 abaixo, das palavras representativas do conteúdo do trabalho, isto é, 
 palavras-chave e/ou descritores, conforme a NBR 6028. (\ldots) As 
 palavras-chave devem figurar logo abaixo do resumo, antecedidas da expressão 
 Palavras-chave:, separadas entre si por ponto e finalizadas também por ponto.
 
 \vspace{\onelineskip}
 
 \noindent
 \textbf{Palavras-chave}: internet das coisas, sistemas embarcados.
\end{resumoumacoluna}


% resumo em inglês
\renewcommand{\resumoname}{Abstract}
\begin{resumoumacoluna}
 \begin{otherlanguage*}{english}
   According to ABNT NBR 6022:2018, an abstract in foreign language is optional.

   \vspace{\onelineskip}
 
   \noindent
   \textbf{Keywords}: 
 \end{otherlanguage*}  
\end{resumoumacoluna}

%\cleardoublepage
%\pdfbookmark[0]{\contentsname}{toc}
%\tableofcontents*
%\cleardoublepage
%\listoffigures
%\cleardoublepage
%\listoftables
\cleardoublepage

%]  				% FIM DE ARTIGO EM DUAS COLUNAS
% ---

% ----------------------------------------------------------
% ELEMENTOS TEXTUAIS
% ----------------------------------------------------------
\textual
%\setcounter{page}{1}
%\pagenumbering{arabic}
% ----------------------------------------------------------
% Introdução
% ----------------------------------------------------------
\section{Introdução}
Tendo em vista a crescente popularização da internet, melhorias nos protocolos de comunicação e a maior capacidade do hardware, abre-se espaço para integrar a rede diversos dispositivos ou ``coisas'', até então nunca imaginados, tais como geladeiras, televisores, entre outros. Esse conceito é chamado de internet das coisas, ``\textit{internet of things}'' (IOT), aplicando-o no contexto agrícola, observa-se um grande potencial.

Observando esse potencial, porém notando as muitas ideias, métodos e sistemas diferentes que realizam funções similares, faz-se necessário que se analise o que o mundo tem feito para implementar sistemas IOT em áreas agrícolas.

\section{Material e Método}\label{Material e Método}
\section{Plataformas de tomada de decisão}\label{Plataformas de tomada de decisão}%artigos 1 e 2
Em relação às plataformas de tomada de decisão segundo a pesquisa de \citeauthor{1} até onde sabe-se no ano de \citeyear{1} não havia nenhuma plataforma ou \textit{framework} que baseando-se na temperatura do solo fornecesse dados dinâmicos para informar em tempo real decisões agronômicas dependentes do solo, tal como momento do plantio, irrigação entre outros.

Considerando essa lacuna do conhecimento, desenvolveu-se uma ferramenta de suporte à decisão da temperatura do solo, ``\textit{temperature decision support tool}'', seguindo uma metodologia de cinco passos:
\begin{enumerate}
  \item Comparação dos dados climáticos e ambientais, conseguindo a variabilidade em larga escala da temperatura do solo.

  \item Comparação das temperaturas médias em séries históricas.
  \item Comparar variáveis preditoras de clima com as medições realizadas, obtendo previsões.
  \item Transformar o sistema para funcionar em tempo real.
  \item Investigações a longo prazo.
\end{enumerate}

Com o objetivo de aumentar a precisão utiliza-se nove variáveis preditoras de clima, sendo elas: temperatura máxima, radiação diária, diferença entre temperatura máxima e mínima, taxa pluviométrica, latitude, elevação, conteúdo de água do solo, difusão térmica do solo, dia do ano.

Este sistema foi desenvolvido Shiny (R), e possui uma alta taxa de predição de 92\% de validação cruzada R$_{2}$, RMSE = 1,91, tendência percentual = $-$ 0,01.
Com essas taxas de predição pode-se auxiliar os agricultores de algodão a realizarem o plantio no momento certo, classificando o solo como ``bom'' após três dias com temperaturas acima de 14\textdegree C, dessa maneira após utilizar dados de variáveis preditoras supracitadas recomenda-se ou não o plantio.

Outro sistema de apoio à decisão desenvolvido por~\citeauthor{2}, com o objetivo de monitorar vinhedos para encontrar e tratar ``míldio'' (mofo).
O míldio é uma doença fúngica causada pela \textit{plasmopara viticola oomycete}, doença essa que causa muito prejuízo nas vinícolas\cite{2}. Doença essa muito estudada, sendo assim nessa pesquisa baseando-se nos modelos já existentes de detecção a acompanhamento do fungo, buscou-se a automatização dos mesmos.

O principal modelo utilizado para descobrir qual o momento mais propício de aparecimento do míldio foi o de Goidanich\cite{detectando_milidio}, também chamado de ``regra dos três dez'' pois quando a temperatura média ultrapassa 10\textdegree C, a germinação ultrapassa os 10 cm e o volume de chuva superior a 10 mm, este é o momento propício para uma primeira contaminação da vinha.
Sabendo desses dados expostos por Goidanich\cite{detectando_milidio} percebe-se a necessidade de monitorar três fenômenos, sendo eles: temperatura, umidade e índice pluviométrico, sendo assim precisando de três sensores.

Para desenvolver esse sistema o autor utilizou o sistema ``\textit{Sense Our Environment}'' (SEnviro), uma plataforma que utilizando-se de software e hardware livres visa baixar o custo e aumentar a eficiência energética dos sensores\cite{2}, esse \textit{framework} base da pesquisa tem suas funcionalidades melhor explicado em outra pesquisa\cite{SEnviro} do mesmo autor, e como já citado nessa seção possui todos os códigos fonte disponível ao público no github\cite{SEnviro_Github}.

Em relação a hardware utilizou-se os sensores de temperatura, umidade e pluviométrico, além de uma bateria alimentada por um painel solar, para controlar e enviar os dados utilizados no sistema utilizou-se uma placa Arduino LinkIt ONE, e para comunicação O módulo ``\textit{Sensor Data Management}'' (SDM).

Com a estacão pronta iniciou-se os testes, detectando de maneira precisa em 96,9\% dos casos em que houve o alerta de infecção com isso os agricultores puderam aplicar o tratamento de controle de praga apenas quando necessário, reduzindo assim a quantidade de produtos químicos no solo, a quantidade de horas gastas para verificação e correção de doenças dentro do vinhedo bem como os custos da plantação.

\section{Considerações Finais}
% ----------------------------------------------------------
% ELEMENTOS PÓS-TEXTUAIS
% ----------------------------------------------------------
\postextual

% ----------------------------------------------------------
% Referências bibliográficas
% ----------------------------------------------------------
\cleardoublepage
\bibliography{ref}

% ----------------------------------------------------------
% Glossário
% ----------------------------------------------------------
%
% Há diversas soluções prontas para glossário em LaTeX. 
% Consulte o manual do abnTeX2 para obter sugestões.
%
%\glossary

\end{document}
