%% abtex2-modelo-artigo.tex, v-1.9.7 laurocesar
%% Copyright 2012-2018 by abnTeX2 group at http://www.abntex.net.br/ 
%%
%% This work may be distributed and/or modified under the
%% conditions of the LaTeX Project Public License, either version 1.3
%% of this license or (at your option) any later version.
%% The latest version of this license is in
%%   http://www.latex-project.org/lppl.txt
%% and version 1.3 or later is part of all distributions of LaTeX
%% version 2005/12/01 or later.
%%
%% This work has the LPPL maintenance status `maintained'.
%% 
%% The Current Maintainer of this work is the abnTeX2 team, led
%% by Lauro César Araujo. Further information are available on 
%% http://www.abntex.net.br/
%%
%% This work consists of the files abntex2-modelo-artigo.tex and
%% abntex2-modelo-references.bib
%%

% ------------------------------------------------------------------------
% ------------------------------------------------------------------------
% abnTeX2: Modelo de Artigo Acadêmico em conformidade com
% ABNT NBR 6022:2018: Informação e documentação - Artigo em publicação 
% periódica científica - Apresentação
% ------------------------------------------------------------------------
% ------------------------------------------------------------------------

\documentclass[
	% -- opções da classe memoir --
	article,			% indica que é um artigo acadêmico
	12pt,				% tamanho da fonte
	oneside,			% para impressão apenas no recto. Oposto a twoside
	a4paper,			% tamanho do papel. 
	% -- opções da classe abntex2 --
	%chapter=TITLE,		% títulos de capítulos convertidos em letras maiúsculas
	%section=TITLE,		% títulos de seções convertidos em letras maiúsculas
	%subsection=TITLE,	% títulos de subseções convertidos em letras maiúsculas
	%subsubsection=TITLE % títulos de subsubseções convertidos em letras maiúsculas
	% -- opções do pacote babel --
	english,			% idioma adicional para hifenização
	brazil,				% o último idioma é o principal do documento
	sumario=tradicional
	]{abntex2}


% ---
% PACOTES
% ---

% ---
% Pacotes fundamentais 
% ---
\usepackage{lmodern}			% Usa a fonte Latin Modern
\usepackage[T1]{fontenc}		% Selecao de codigos de fonte.
\usepackage[utf8]{inputenc}		% Codificacao do documento (conversão automática dos acentos)
\usepackage{indentfirst}		% Indenta o primeiro parágrafo de cada seção.
\usepackage{nomencl} 			% Lista de simbolos
\usepackage{color}				% Controle das cores
\usepackage{graphicx}			% Inclusão de gráficos
\usepackage{microtype} 			% para melhorias de justificação

% ---
% Pacotes adicionais, usados apenas no âmbito do Modelo Canônico do abnteX2
% ---
% ---
% Pacotes de citações
% ---
%\usepackage[brazilian,hyperpageref]{backref}	 % Paginas com as citações na bibl
%\usepackage[alf]{abntex2cite}	% Citações padrão ABNT

\usepackage[brazilian,hyperpageref]{backref}	 % Paginas com as citações na bibl
%\usepackage[alf]{abntex2cite}	% Citações padrão ABNT
\usepackage[num,overcite,abnt-emphasize=bf]{abntex2cite}	% Citações padrão ABNT
%\usepackage[num]{abntex2cite}	% Citações padrão ABNT
%\citebrackets()
\citebrackets[]

% ---

% ---
% Configurações do pacote backref
% Usado sem a opção hyperpageref de backref
\renewcommand{\backrefpagesname}{Citado na(s) página(s):~}
% Texto padrão antes do número das páginas
\renewcommand{\backref}{}
% Define os textos da citação
\renewcommand*{\backrefalt}[4]{
	\ifcase #1 %
		Nenhuma citação no texto.%
	\or
		Citado na página #2.%
	\else
		Citado #1 vezes nas páginas #2.%
	\fi}%
% ---

% --- Informações de dados para CAPA e FOLHA DE ROSTO ---
% \titulo{Modelo Canônico de Artigo científico com \abnTeX}
% \tituloestrangeiro{Canonical article template in \abnTeX: optional foreign title}

\titulo{Uso de software livre para estudantes de Ciência da Computação (revisão da literatura, revisão narrativa)}
\tituloestrangeiro{Use of free software for Computer Science students}

\autor{Victor Emanuel Almeida}

\local{FOZ DO IGUAÇU}
\data{\today}

%\instituicao{%
	%\par
	%Universidade do Oeste do Paraná 
	%\par
	%UNIOESTE
%}
\instituicao{Curso de Ciência da Computação, da Universidade Estadual do Oeste do Paraná (UNIOESTE), Campus Foz do Iguaçu-PR, Brasil}

\preambulo{Escrever Preâmbulo}

\tipotrabalho{Trabalho Acadêmico}
% ---

% ---
% Configurações de aparência do PDF final

% alterando o aspecto da cor azul
\definecolor{blue}{RGB}{41,5,195}

% informações do PDF
\makeatletter
\hypersetup{
  %pagebackref=true,
  pdftitle={\@title}, 
  pdfauthor={\@author},
  pdfsubject={software livre},
  pdfcreator={\@author},
  pdfkeywords={software livre},
  colorlinks=true,       		% false: boxed links; true: colored links
  linkcolor=black,          	% color of internal links
  citecolor=blue,        		% color of links to bibliography
  filecolor=magenta,      		% color of file links
  urlcolor=blue,
  bookmarksdepth=4
}
\makeatother
% --- 

% ---
% compila o indice
% ---
\makeindex
% ---

% ---
% Altera as margens padrões
% ---
\setlrmarginsandblock{3cm}{2cm}{*}
\setulmarginsandblock{3cm}{2cm}{*}
\checkandfixthelayout
% ---

% --- 
% Espaçamentos entre linhas e parágrafos 
% --- 

% O tamanho do parágrafo é dado por:
\setlength{\parindent}{1.3cm}

% Controle do espaçamento entre um parágrafo e outro:
%\setlength{\parskip}{0.2cm}  % tente também \onelineskip

% Espaçamento simples
\SingleSpacing


% ----
% Início do documento
% ----
\begin{document}
\pagenumbering{roman}
% Seleciona o idioma do documento (conforme pacotes do babel)
%\selectlanguage{english}
\selectlanguage{brazil}

% Retira espaço extra obsoleto entre as frases.
\frenchspacing 

% ----------------------------------------------------------
% ELEMENTOS PRÉ-TEXTUAIS
% ----------------------------------------------------------

%---
%
% Se desejar escrever o artigo em duas colunas, descomente a linha abaixo
% e a linha com o texto ``FIM DE ARTIGO EM DUAS COLUNAS''.
%\twocolumn[    		% INICIO DE ARTIGO EM DUAS COLUNAS
%
%---

% página de titulo principal (obrigatório)
\imprimircapa
%\begin{center}
  %\maketitle
  %\imprimirtitulo
  %\vspace{.5cm}

  %\imprimirinstituicao
%\end{center}

% titulo em outro idioma (opcional)



% resumo em português
\begin{resumoumacoluna}
 Conforme a ABNT NBR 6022:2018, o resumo no idioma do documento é elemento obrigatório. 
 Constituído de uma sequência de frases concisas e objetivas e não de uma 
 simples enumeração de tópicos, não ultrapassando 250 palavras, seguido, logo 
 abaixo, das palavras representativas do conteúdo do trabalho, isto é, 
 palavras-chave e/ou descritores, conforme a NBR 6028. (\ldots) As 
 palavras-chave devem figurar logo abaixo do resumo, antecedidas da expressão 
 Palavras-chave:, separadas entre si por ponto e finalizadas também por ponto.
 
 \vspace{\onelineskip}
 
 \noindent
 \textbf{Palavras-chave}: software livre. vantagens. 
\end{resumoumacoluna}


% resumo em inglês
\renewcommand{\resumoname}{Abstract}
\begin{resumoumacoluna}
 \begin{otherlanguage*}{english}
   According to ABNT NBR 6022:2018, an abstract in foreign language is optional.

   \vspace{\onelineskip}
 
   \noindent
   \textbf{Keywords}: free software. advantages
 \end{otherlanguage*}  
\end{resumoumacoluna}

\cleardoublepage
\pdfbookmark[0]{\contentsname}{toc}
\tableofcontents*
%\cleardoublepage
\listoffigures
%\cleardoublepage
\listoftables
\cleardoublepage

% ]  				% FIM DE ARTIGO EM DUAS COLUNAS
% ---

% ----------------------------------------------------------
% ELEMENTOS TEXTUAIS
% ----------------------------------------------------------
\textual
%\setcounter{page}{1}
\pagenumbering{arabic}
% ----------------------------------------------------------
% Introdução
% ----------------------------------------------------------
\section{Introdução}\label{Introdução}

Essa é a Introdução

Agora cito o~\cite{definicao}, tbm cito o~\cite{uso_de_linux}, e o~\cite{vantagens}.

\cite{site-fsf}

\section{Definição de software livre}\label{Definição de software livre}
Antes de entendermos o motivo de usarmos software livre, seus benefícios e vantagens, precisamos primeiramente entender o que é software livre.

Tendo em vista a definição dada pela pela Free Software Foundation~\cite{site-fsf}, bem como a tradução para o português feita por~\cite{definicao}, software livre é ``o software que pode ser usado, copiado, estudado, modificado e redistribuído sem restrição. A forma usual de um software ser distribuído livremente é sendo acompanhado por uma licença de software livre, e com a disponibilização do seu código-fonte.''

\section{Vantagens do software livre}\label{Vantagens do software livre}

\section{Alternativas software livre para uso dentro da universidade}\label{Alternativas software livre para uso dentro da universidade}

Os softwares citados nas seções abaixo, seguem a filosofia Unix, que teria como um dos princípios ``Faça cada programa fazer uma coisa bem feita''\cite{filosofia_unix}, para diversas áreas do conhecimento.

\subsection{Vim}\label{Editor de texto}

\subsection{Latex}\label{Latex}
\LaTeX

\subsection{Linux}\label{Linux}


\subsection{Logisim}\label{Logisim}
``O Logisim é uma ferramenta educacional para a concepção e a simulação digital de circuitos lógicos''\cite{site_logisim}.

\subsection{Jflap}\label{Jflap}
O Jflap é um ``pacote gráfico com ferramentas que podem ser utilizadas para aprender conceitos básicos de Linguagens formais e autômatos''\cite{site_jflap}.

% ----------------------------------------------------------
% ELEMENTOS PÓS-TEXTUAIS
% ----------------------------------------------------------
\postextual

% ----------------------------------------------------------
% Referências bibliográficas
% ----------------------------------------------------------
\cleardoublepage
\bibliography{referencia}

% ----------------------------------------------------------
% Glossário
% ----------------------------------------------------------
%
% Há diversas soluções prontas para glossário em LaTeX. 
% Consulte o manual do abnTeX2 para obter sugestões.
%
%\glossary

\end{document}

